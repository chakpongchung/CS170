\documentclass[11pt]{article}
\usepackage{amsmath,textcomp,amssymb,geometry,graphicx,tikz}
\usepackage[T1]{fontenc}

\def\Name{Zackery Field}  % Your name
\def\Sec{Di, 103}  % Your GSI's name and discussion section
\def\Login{cs170-fe} % Your login
\def\Homework{11}%Number of Homework
\def\Session{Fall 2013}

\title{CS170  Fall 2013 Solutions to Homework 11}
\author{\Name, section \Sec, \texttt{\Login}}
\markboth{CS170 --\Session\  Homework \Homework\ \Name, section \Sec}
{CS170--\Session\ Homework \Homework\ \Name, section \Sec, \texttt{\Login}}
\pagestyle{myheadings}

\begin{document}
\maketitle

\section{[10 pts.] Mindbending Logic} 

Show that if Horn SAT is not NP-complete, then P$\neq$NP. 
(Hint: Consider the contrapositive.)

Following the hint, let the contrapositive be; If P$=$NP, then the Horn SAT 
is NP-complete. It is proven in the text that Horn SAT is in P. Since Horn SAT
is in P, if P$=$NP, then Horn SAT is also NP-complete. 

\label{pg:end-of-p1}
% Make sure that the solution here does not exceed one page here. If
% it does, use the extra space for this problem at the end.  
%
% Comment out the next line if you are NOT using the extra space
% \paragraph{} \emph{Continued on Page \pageref{pg:p1-continuation}}

\newpage

%%Do NOT remove/comment the next line
\pagestyle{plain}
%%It makes sure your name appears only on the first page

\section{[20 pts.] Proving NP-completeness by Generalization}

For each of the problems below, prove that it is NP-complete by showing that 
it is a \emph{generalization} of an NP-complete problem.

\begin{itemize}

\item[(a)] Subgraph Isopmorphism: Given as input two undirected graphs $G$
  and $H$, determine whether $G$ is a subgraph of $H$ (that is, whether by 
  deleting certain vertices and edges of $H$, we obtain a graph that is,
  up to renaming vertices, identical to $G$), and if so, return the
  corresponding mapping of $V(G)$ into $V(H)$.

  This is a generalization of the clique problem. The clique problem
  seeks a clique of size $k$.
  To have the subgraph isomorphism output the clique($G,k$) of some 
  we can construct a graph $T$ of size $k$ that is complete. Then $T$ 
  will be a subgraph of $G$ if there is a clique of size $k$ in $G$. Since 
  subgraph isomorphism is a generalization of the clique problem, 
  subgraph isomorphism is NP-complete.

\item[(b)] Longest Path: Given a graph $G$ and an integer $g$, find in $G$ 
  a simple path of length $g$.

  This is a generalization of the Rudrata Path problem. Take some
  graph $G=(V,E)$. The Rudrata Path problem takes this graph $G$ and 
  finds some simple path of length $g=|V|$, xor that none exists.
  (If $G$ is weighted, then you 
  can easily make it unweighted.) So for any Rudrata path problem Rudrata$(G)$,
  we can form a Longest Path problem, Longest$(G,g=|V|)$. Since the 
  Rudrata path problem is NP-complete, the
  longest path problem is NP-complete.

\item[(c)] Max SAT: Given a CNF formula and an integer $g$, find a truth 
  assignment that satisfies at least $g$ clauses. 

  This problem is a generalization of the SAT problem.  Take some instance
  of a SAT problem $I=(CNF)$. Since SAT is attempting to discover any 
  satisfying set or return that none exists, 
  we can set $g=0$ and run Max\_SAT($I=(CNF),g=0$). Since SAT is NP-complete, 
  Max SAT is also NP-complete.

\item[(d)] Sparse Subgraph: Given a graph and two integers $a$ and $b$, 
  find a set of $a$ vertices of $G$ such that there are at most $b$ edges
  between them.

  This is a generalization of the vertex cover problem. Let $V=(G,a)$ be 
  a vertex cover problem instance, where $G=(V,E)$ is some graph, and
  $a$ is the maximum number of vertices in your cover.
  This vertex cover 
  problem is equivalent to a Sparse Subgraph problem where $b=|E|$: 
  Sparse\_SubGraph($V=(G,a),b=|E|$). Since vertex cover is NP-complete,
  the sparse subgraph problem is also NP-complete.
  
\end{itemize}


\label{pg:end-of-p2}
% Make sure that the solution here does not exceed one page here. If
% it does, use the extra space for this problem at the end.  
%
% Comment out the next line if you are NOT using the extra space
% \paragraph{} \emph{Continued on Page \pageref{pg:p2-continuation}}

\newpage

\section{[20 pts.] Subset Product}

Similar to Subset Sum, in the Subset Product problem you are given a set of 
integers $A = \{a_1, a_2,\ldots, a_n\}$ and a goal $g$, and your job is to 
determine whether there is a subset $S\subseteq A$ such that the product of 
the numbers in $S$ is exactly $g$.

\begin{itemize}

\item[(a)] Prove that Subset Product is in NP.

  Given some solution subset $S$ of a Subset\_Problem($A,g$). Find the 
  product of the subset of integers in $S$ and determine if $prod(S)=g$. 
  Since the multiplication of a finite set of terms is polynomial,
  verification of a solution to a Subset Problem is polynomial in
  $S$, and the Subset Product is in NP.

\item[(b)] Point out the flaw in the given proof of NP-completeness.

  In the proof, they create a set of numbers $A'=\{2^{a_1},2^{a_2},\cdots,2^{a_n}\}$.
  The process of creating this set is exponential in each term of 
  $A={a_1,a_2,\ldots,a_n}$. This is an invalidation of the proof because 
  a reduction argument requires that the function $f(I)$
  (for some problem instance I) be polynomial in $I$. 

\item[(c)] One of the first problems to be proved NP-complete is 
  \emph{Exact Cover}, which is like Set Cover, except that the chosen
  subsets have to be \emph{disjoint}. Prove that Subset Product is NP-complete
  by a reduction from Exact Cover. Make sure you prove that you do not make 
  the same mistake as the proof in (b). (Hint: You may assume without proof 
  that the first $n$ prime numbers can be calculated in time polynomial in 
  $n$ , and the $n$th prime number is $O(n\log n)$.) 

  We are given an instance $I$ of a set cover problem with subsets $\{S_1,
  S_2,\ldots,S_n\}$. We can assign each of the elements in these sets 
  a prime number. Such that two elements that are equal (but in different 
  sets) are given the same prime number. 
  The smallest number in the sets is assigned $2$, the 
  next element $3$, all the way up to the number of distinct inputs, $n$. 
  The goal 
  in the reduced subset product problem is the product of the first $n$ prime
  numbers. If a subset product is found to equal this goal value, then there 
  exists a disjoint collection of subsets that covers the initial exact cover
  instance $I$.
  
  
  
\label{pg:end-of-p3}

%% Make sure that the solution here does not exceed one page here. If
%% it does, use the extra space for this problem at the end.  
%% 
%% Comment out the next line if you are NOT using the extra space
%%\paragraph{} \emph{Continued on Page \pageref{pg:p3-continuation}}

%\newpage
%  
%\section*{Extra space for Problem 1}
%\emph{Continued from Page \pageref{pg:end-of-p1}}
%
%
%\label{pg:p1-continuation}
 
 
%\newpage
%%Comment out the above three lines if you are not using extraspace
%%for this problem.
 
 
 %\section*{Extra space for Problem 2}
 %\emph{Continued from Page \pageref{pg:end-of-p2}}
  
 
 %\label{pg:p2-continuation}
 %\newpage

 %%Comment out the above three lines if you are not using extra space
 %%for this problem.
 
 %\section*{Extra space for Problem 3}
 %\emph{Continued from Page \pageref{pg:end-of-p3}}
 
%\label{pg:p3-continuation}

% %%\newpage


\end{document}

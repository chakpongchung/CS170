\documentclass[11pt]{article}
\usepackage{amsmath,textcomp,amssymb,geometry,graphicx,tikz}
\usepackage{algorithm}
\usepackage{algpseudocode}
\usepackage[T1]{fontenc}

\def\Name{Zackery Field}  % Your name
\def\Sec{Di, 103}  % Your GSI's name and discussion section
\def\Login{cs170-fe} % Your login
\def\Homework{12}%Number of Homework
\def\Session{Fall 2013}

\title{CS170  Fall 2013 Solutions to Homework 12}
\author{\Name, section \Sec, \texttt{\Login}}
\markboth{CS170 --\Session\  Homework \Homework\ \Name, section \Sec}
{CS170--\Session\ Homework \Homework\ \Name, section \Sec, \texttt{\Login}}
\pagestyle{myheadings}

\begin{document}
\maketitle

\section{[10 pts.] Making DAGs is hard}

Consider the search problem Max-Acyclic-Induced-Subgraph: 

INPUT: A \emph{directed} graph $G=(V,E)$, and a positive integer $k$.

OUTPUT: A subset $S\subseteq V$ of size $k$ such that the graph $G_S$ obtained
from $G$ by keeping exactly those edges whose endpoints are in $S$ is a 
DAG.

Show that Max-Acyclic-Induced-Subgraph is NP-complete.

\begin{itemize}

\item Show that the Max-Acyclic-Induced-Subgraph (MAIS) problem is in NP.

  Given an output subset $S \subseteq V$ we can run a DFS on $G_S$ and determine if
  there are any cycles. If there are no cycles, then a DAG of proper size ($k$),
  was constructed. If there is a cycle then $G_S$ is not a DAG. The graph $G_S$ 
  can be constructed from $G$ in polynomial time by iterating through the vertices in
  $S$ and including only edges whose endpoints $u$ and $v$ are both in $S$. The
  cycle check on $G_S$ can be done by DFS, which runs in linear time. Therefore,
  a solution to a MAIS problem can be checked in polynomial time, which implies that
  MAIS is in NP.
  
\item Reduction:  {\bf Independent Set  $\rightarrow$ MAIS},
  to show NP-completeness.
  
  Let $T=(G,k)$ be some instance of an independent set problem, where
  $G=(V,E)$ is the undirected input graph, and $k$ is the size of the desired independent subset 
  of vertices $V$. If there is no independent subset of size $k$, then the algorithm
  should output that none exists. If a subset does exist, then it should output that
  subset. 
  We first define a function $f(T)$ 
  that takes as input an undirected graph $G$ and desired set-size $k$,
  and turns $G$ into a directed graph $G'$. This directed graph, and its associated
  $k$ value, form the inputs of a MAIS problem. 

  Construction of $G=(V,E)\rightarrow G'=(V,E')$: 

  For each of the undirected edges $e = \{u-v\}, \forall e\in E$, replace that edge
  with two directed edges $\{u \rightarrow v\}$, and $\{v \rightarrow u\}$.  
   
  
\end{itemize}

\label{pg:end-of-p1}
% Make sure that the solution here does not exceed one page here. If
% it does, use the extra space for this problem at the end.  
%
% Comment out the next line if you are NOT using the extra space
\paragraph{} \emph{Continued on Page \pageref{pg:p1-continuation}}

\newpage

%%Do NOT remove/comment the next line
\pagestyle{plain}
%%It makes sure your name appears only on the first page

\section{[10 pts.] Reductions redux}

Show that you can not hope to do much better than insertion 
or deletion worst case complexity $\Omega(n \log n)$, where $n$ is the 
number of elements in queue. Prove that in the "comparison model"
, there cannot exist a priority queue implementation in which
both Insert and Delete-Min operations have worst case complexity $O(1)$. 
 
Lemma: Comparison Sorting $n$ elements takes $O(n\log n)$ time. 

Assume, for the sake of contradiction, that we have some priority queue whose
insert and delete-min updates both take $O(1)$ time. 
We have some set of $n$ items in our priority queue, and 
each of these items ($i$) can be compared to any other item ($j$) to reveal a
relationship, either $i<j$ or $i>j$ in $O(1)$ time. 
Without loss of generality, we can implement the priority queue with a sorted
array whose delete-min update takes $O(1)$. This running time is perfectly valid
since we can lookup the min entry in $O(1)$ time. As the assumption states, the 
insert update must also take $O(1)$ time. In terms of the sorted array, this means
that there is some update procedure by which you can insert an element into a sorted
array in $O(1)$ time. If this is the case, then creating a sorted array for
$n$ unsorted items would only take $O(n)$ time, a contradiction. Therefore, 
there cannot be an implementation of a priority queue that has $O(1)$ insert and
delete-min updates in the Comparison-Model. 

\label{pg:end-of-p2}
% Make sure that the solution here does not exceed one page here. If
% it does, use the extra space for this problem at the end.  
%
% Comment out the next line if you are NOT using the extra space
%\paragraph{} \emph{Continued on Page \pageref{pg:p2-continuation}}

\newpage

\section{[10 pts.] Finding Zero(s)} 

Consider the problem Integer-Zeros. 

INPUT: A multivariate polynomial $P(x_1,x_2,x_3,\ldots,x_n)$ with integer
coefficients, specified as a sum of monomials. 

OUTPUT: Integers $a_1,a_2,\ldots,a_n$ such that $P(a_1,a_2,a_3,\ldots,a_n)=0$.

Show that 3-SAT reduces in polynomial time to Integer-Zeros. 

  
%Following the hint on piazza, I will define the Many-Integer-Zeros problem 
%to be: Given a set of polynomials $\mathbb{P} = \{P_1,P_2,\ldots,P_n\}$, where each
%
%$P_i(a_1,a_2,\ldots,a_n)$. 
%Find a set of integers $\{a_1,a_2,\ldots,a_n\}$ such that $P_1=P_2=\cdots=P_n=0$.
%
Assuming that CNF is provided, we are given a 3-SAT problem statement $S$
of the form: 

$S = \{(x_1 \vee x_2 \vee \bar{x_3})
\wedge (x_1 \vee \bar{x_i} \vee \bar{x_n})
\wedge \cdots
\wedge (x_n \vee x_2 \vee x_{i+1})\}$

Let $f_{IZ}(S)$ be the function that formulates an Integer-Zeros problem 
from a 3-SAT problem. Given a 3-clause $(x_1 \vee x_2 \vee x_3)$, 
this function will construct
a polynomial $P_i(x_1x_2x_3)=x_1x_2x_3$.  
Since all of the relations within a 3-clause are OR and none of the clauses are
negated, 
we can simply multiply the 3-clause into one monomial. This way, whenever one of the
$x_i$ clauses is zero, the entire polynomial will be zero, which is equivalent to
the or relation where $0\equiv$TRUE. If there are one or more single clause ($\bar{x}$) 
in the 3-clause that is/are negated, then this can be represented by inserting $(1-x)$ into the monomial.
Here is one such example:  $(x_1 \vee \bar{x_2} \vee \bar{x_3})$, $P_i(x_1x_2x_3)=x_1(1-x_2)(1-x_3)$.
The only way to make a $1-x$ negation clause true (zero) is to set $x=1$, thus 
$1\equiv$FALSE.
The AND relations
between each of the 3-clauses is represented by the summation of all of the individual
3-clause monomials; $(x_1x_2x_3)+\cdots+(x_1(1-x_2)(1-x_3)) = 0$. This summation is equivalent to 
the boolean AND relation, because if any single monomial is nonzero, then the 
entire sum will be nonzero.

%Possible example, time permitting

Now that we have defined a polynomial time algorithm for constructing an Integer-Zeros problem, 
we must determine if this problem is equivalent. Each monomial is constructed s.t. any one
zero value for some $x_i$, or some unit value for some $(1-x_i)$ will be sufficient to make the entire term zero.
This is consistent with the 3-clauses, because in CNF the relation between each of the clauses in a 3-clause 
is OR. The AND between each 3-clause is represented by the summation of the individual monomials, as stated
above. This summation construction is sufficient to ensure that all monomials are satisfied in order for the
entire polynomial to be satisfied.

The Integer-Zeros problem will either output a set of inputs 
$\{a_1,a_2,\cdots,a_n\}$ which map to $P(a_1,a_2,\cdots,a_n)=0$ for some $P(x_1,x_2,\cdots,x_n)$,
or it will output that no such inputs exist. This set of inputs, by construction of $P$, 
will either be zero, or one, corresponding to TRUE and FALSE, respectively. 
A function $h(O)$ that maps this returned set of inputs back to a solution to the
original 3-SAT will simply look at the values of the individual clauses $x_i$, then assign
values based on the relation:
$x_i=$TRUE if $a_i=0$, and $x_i=$FALSE if $a_i=1$. 

Note: Since the polynomial $P$ is meant to be represented as the sum of polynomials
in order to ensure that each unit clause $x_i$ is given a value in the set of output inputs 
(rather than just the compounded 3-clauses) insert for each $x_i$ one monomial:
$x_i(1-x_i)$. This will also ensure that each value in the output set of inputs is either
zero or one. 

\label{pg:end-of-p3}

% Make sure that the solution here does not exceed one page here. If
% it does, use the extra space for this problem at the end.  
% 
% Comment out the next line if you are NOT using the extra space
%\paragraph{} \emph{Continued on Page \pageref{pg:p3-continuation}}

\newpage

\section{[10 pts] Approximately independent}

Consider the following algorithm for finding an independent set in an 
undirected graph $G=(V,E)$.

\begin{algorithm}
  \caption{Finding a large independent set}
  \begin{algorithmic}[1]
    \State{{\bf Input} A graph $G=(V,E)$}
    \State{{\bf Output} A subset $I\subseteq V$}
    \While{$V \neq \emptyset$}
    \State {Let $v$ be the vertex with the \emph{smallest} degree in $G'=(V,E)$}
    \State{$I \leftarrow I\cup\{v\}$}
    \State{Let $S$ be the set of neighbors of $v$ in $G'$}
    \State{$V \leftarrow V - (S \cup \{v\})$}
    \State{$E \leftarrow E - \{e|e \mbox{ is incident on a vertex in }S$}
    \EndWhile
  \end{algorithmic}
\end{algorithm}

\begin{itemize}

\item[(a)] Show that the output of algorithm $1$ is an independent set. 

  Let $G=(V,E)$ be some problem instance of the algorithm, and let $I\subseteq V$
  be the solution that is output.
  
  Assume, for the sake of contradiction, that $I$ is not an independent set. 
  By the definition of independence, for some two vertices $\{u,v\}\in I$, $u$ and 
  $v$ share an edge; $u$ and $v$ are neighbors. 
  
  For some iteration of the algorithm $v$ (or $u$, arbitrary choice) is the vertex of
  smallest degree in $G'$ (line $4$). $v$ is then added to $I$ in line $5$.
  When the set $S$ of neighbors of $v$ is constructed
  in line $6$, $u\in S$, by definition. Then, (line $7$) all edges $\in S$ and $v$ are removed
  from the list of iterable vertices $V$. This step ensures that $u$ will never be
  chosen at a later iteration. In the last line (line $8$) all edges incident on a
  vertex in $S$, including the edge $u - v$, are removed from the set of edges $E$. This 
  is a contradiction, since the assumption at the start was that $u$ and $v$ are neighbors, 
  but the edge $u - v$ that defines them as such is removed, during line $8$. Therefore,
  $I$ is an independent set. 

\end{itemize}

\label{pg:end-of-p4}

\paragraph{} \emph{Continued on Page \pageref{pg:p4-continuation}}


\newpage

\section{[10 pts.] Coin Tosses vs Local Search}

We are given a graph $G = (V,E)$ and our goal is to produce a partition
$(A,V-A)$ of $V$ such that the number of edges crossing the cut $(A,V-A)$
is as large as possible. Let $C$ be the \emph{maximum} size of such a cut. We
consider two approaches for solving this problem:

{\bf Devil-May-Care} For each vertex $v \in V$, we toss a fair coin. If the coin
comes up heads, we put $v$ in $A$, else we don't put it in $A$. 

{\bf Prudence-Is-Us} Given a subset $S\subseteq V$ and a vertex $v \in V$, let
$N(v,S)$ denote the number of neighbors of $v$ in $S$ (not including $v$, 
in case $v \in S$). We then use algorithm 2.

\begin{algorithm}
  \caption{Finding the large cuts}
  \begin{algorithmic}
    \State{{\bf Input} A \emph{graph} $G=(V,E)$}
    \State{{\bf Output} A \emph{subset} $A \subseteq V$}
    \State{$A \leftarrow \emptyset$; flag $\leftarrow$ {\bf true}}
    \Repeat
    \State{flag $\leftarrow$ {\bf false}}
    \If {$\exists v \in V - A$ \emph{such that} $N(v,A) < N(v,V-A)$}
      \State{$A \leftarrow A\cup \{v\}$}
      \State{flag $\leftarrow$ {\bf true}}
    \ElsIf{$\exists v\in A$ \emph{such that} $N(v,A) > N(v,V-A)$}
      \State{$A \leftarrow A - v$}
      \State{flag $\leftarrow$ {\bf true}}
    \EndIf
    \Until{ flag $\neq$ {\bf true};}
  \end{algorithmic}
\end{algorithm}

\begin{itemize}
  \item[(a)] Let $(A,V-A)$ be the partition produced by the first approach.
    Show that the expected number of edges crossing the cut is at least $C/2$.


  \item[(b)] Show that there are at most $|E|+1$ iterations of the loop at
    line 4 in algorithm 2.


  \item[(c)] Show that if $A$ is the set output by algorithm 2, then the number 
    of edges crossing $(A,V-A)$ is at least $C/2$.
\end{itemize}

\label{pg:end-of-p5}

%\paragraph{} \emph{Continued on Page \pageref{pg:p5-continuation}}

\newpage
  
\section*{Extra space for Problem 1}
\emph{Continued from Page \pageref{pg:end-of-p1}}


\label{pg:p1-continuation}

%\GRAPHIMAGE
  The MAIS problem $(G',k)$ will output either a subset of vertices $S$ that constitute
  a DAG, or it will output that no subset of size $k$ exists. This solution subset
  requires no post-processing function $h(S)$. Any output of the MAIS problem 
  ,with the pre-processing function $f(T)$, will output an independent set. 
  
  Proof: 

  Note that one cannot choose two adjacent vertices, $\{u,v\}$,
  to be placed in the DAG subset $S$ of any MAIS problem. 
  This is because the particular
  construction chosen for the directed graph input forms a cycle between any two
  adjacent vertices. 

  Remember that a DAG is defined as having no cycles, so for this problem
  their has to be some equivalence between the cycles in graph $G'$ and 
  the non-independence of the vertices in these cycles. In other words, 
  a subset of vertices of $G'$ (and their component edges) is acyclic if and only if the subset
  is also an independent set on $G$. 
  
  The trivial portion of this equivalence is; Any independent set is acyclic. 
  In the graph $G'$, if no two vertices are pairwise adjacent in the subset (independent),
  then there is no cycle that can be created, because none of the vertices are connected. 
  (As the problem states, an edge is only considered if \emph{both} of its endpoints are in the subset)
  
  The reverse is also true; Any acyclic subset of vertices of $G'$ is independent on $G$.
  Let $R$ be some acyclic subset of vertices. Assume for the sake of contradiction that $R$ is not independent. 
  Then there exists two vertices $u$, and $v$ that are adjacent. But by construction, these two 
  vertices create a cycle because both  $\{u \rightarrow v\}$, and $\{v \rightarrow u\}$ exist, a contradiction.
  Therefore, any acyclic subset of $G'$ is also an independent set on $G$. This shows
  that the output to the MAIS problem on $\{G',k\}$ will output an independent subset of size $k$,
  if and only if one exists.
   
\newpage
% 
%\section*{Extra space for Problem 2%}
%\emph{Continued from Page \pageref{pg:end-of-p2}%}
%%
%% 
%\label{pg:p2-continuation}
%\newpage
 
%\section*{Extra space for Problem 3}
%\emph{Continued from Page \pageref{pg:end-of-p3}}
% 
%\label{pg:p3-continuation}
%
%\newpage

\section*{Extra space for Problem 4}
\emph{Continued from Page \pageref{pg:end-of-p4}}

\label{pg:p4-continuation}

\begin{itemize}

\item[(b)]  Show that if the maximum degree of a vertex in $G$ is $\Delta$, and the
  size of the maximum independent set is $\alpha$, then we have
  $|I|\leq \frac{\alpha}{\Delta+1}$.

  Start by rearranging the inequality to get: $\Delta+1 \leq \frac{\alpha}{|I|}$
  For a lower bound on $\Delta$, if we take a single vertex graph that has no self loops, 
  then $\Delta =0$. This shows that $\alpha = |I|$ is possible, if and only if $\Delta =0$.
  An upper bound on $\Delta$ would be some completey connected graph, where if $n$
  is the number of vertices then $\Delta = n-1$. In this case, we can intuit that
  $\alpha = 1$, and the inequality becomes 
  $(n-1)+1 \leq \frac{1}{|I|} \Rightarrow |I|n \leq 1 \Rightarrow |I|=0$. But this
  is a contradiction since we know that on the first iteration of the algorithm 
  at least one vertex is added to $|I|$. Therefore, the inequality does not hold. 
 

\end{itemize}


\newpage

\section*{Extra space for Problem 5}
\emph{Continued from Page \pageref{pg:end-of-p5}}


\end{document}

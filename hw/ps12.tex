\documentclass[11pt]{article}
\usepackage{amsmath,textcomp,amssymb,geometry,graphicx,tikz}
\usepackage{algorithm,algorithmic}
\usepackage[T1]{fontenc}

\def\Name{Zackery Field}  % Your name
\def\Sec{Di, 103}  % Your GSI's name and discussion section
\def\Login{cs170-fe} % Your login
\def\Homework{12}%Number of Homework
\def\Session{Fall 2013}

\title{CS170  Fall 2013 Solutions to Homework 12}
\author{\Name, section \Sec, \texttt{\Login}}
\markboth{CS170 --\Session\  Homework \Homework\ \Name, section \Sec}
{CS170--\Session\ Homework \Homework\ \Name, section \Sec, \texttt{\Login}}
\pagestyle{myheadings}

\begin{document}
\maketitle

\section{[10 pts.] Making DAGs is hard}

Consider the search problem Max-Acyclic-Induced-Subgraph: 

INPUT: A \emph{directed} graph $G=(V,E)$, and a positive integer $k$.

OUTPUT: A subset $S\subseteq V$ of size $k$ such that the graph $G_S$ obtained
from $G$ by keeping exactly those edges whose endpoints are in $S$ is a 
DAG.

Show that Max-Acyclic-Induced-Subgraph is NP-complete.

\begin{itemize}

\item Show that the Max-Acyclic-Induced-Subgraph (MAIS) problem is in NP.

\item Reduce to Independent Set / Vertex Cover / Clique

\end{itemize}

\label{pg:end-of-p1}
% Make sure that the solution here does not exceed one page here. If
% it does, use the extra space for this problem at the end.  
%
% Comment out the next line if you are NOT using the extra space
\paragraph{} \emph{Continued on Page \pageref{pg:p1-continuation}}

\newpage

%%Do NOT remove/comment the next line
\pagestyle{plain}
%%It makes sure your name appears only on the first page

\section{[10 pts.] Reductions redux}

Show that you can not hope to do much better than insertion 
or deletion worst case complexity $Omega(\log n)$, where $n$ is the 
number of elements in queue. Prove that in the "comparison model"
, there cannot exist a priority queue implementation in which
both Insert and Delete-Min operations have worst case complexity $O(1)$. 
Hint: Do an appropriate reduction from sorting in the comparison mode. 
 
\label{pg:end-of-p2}
% Make sure that the solution here does not exceed one page here. If
% it does, use the extra space for this problem at the end.  
%
% Comment out the next line if you are NOT using the extra space
\paragraph{} \emph{Continued on Page \pageref{pg:p2-continuation}}

\newpage

\section{[10 pts.] Finding Zero(s)} 

Consider the problem Integer-Zeros. 

INPUT: A multivariate polynomial $P(x_1,x_2,x_3,\ldots,x_n)$ with integer
coefficients, specified as a sum of monomials. 

OUTPUT: Integers $a_1,a_2,\ldots,a_n$ such that $P(a_1,a_2,a_3,\ldots,a_n)=0$.

Show that 3-SAT reduces in polynomial time to Integer-Zeros. (INTEGER ZEROS
IS NOT IN NP)   
  
\label{pg:end-of-p3}

% Make sure that the solution here does not exceed one page here. If
% it does, use the extra space for this problem at the end.  
% 
% Comment out the next line if you are NOT using the extra space
\paragraph{} \emph{Continued on Page \pageref{pg:p3-continuation}}

\newpage

\section{[10 pts] Approximately independent}

Consider the following algorithm for finding an independent set in an 
undirected graph $G=(V,E)$.



\label{pg:end-of-p4}

%\paragraph{} \emph{Continued on Page \pageref{pg:p4-continuation}}

\newpage
  
\section*{Extra space for Problem 1}
\emph{Continued from Page \pageref{pg:end-of-p1}}


\label{pg:p1-continuation}
 
 
\newpage
%Comment out the above three lines if you are not using extraspace
%for this problem.
 
 
\section*{Extra space for Problem 2}
\emph{Continued from Page \pageref{pg:end-of-p2}}

 
\label{pg:p2-continuation}
\newpage

%Comment out the above three lines if you are not using extra space
%for this problem.
 
\section*{Extra space for Problem 3}
\emph{Continued from Page \pageref{pg:end-of-p3}}
 
\label{pg:p3-continuation}

\newpage

\section*{Extra space for Problem 4}
\emph{Continued from Page \pageref{pg:end-of-p4}}


\end{document}

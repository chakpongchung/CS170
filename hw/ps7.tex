\documentclass[11pt]{article}
\usepackage{amsmath,textcomp,amssymb,geometry,graphicx,tikz,cancel}
\usepackage{algpseudocode,algorithm}
\usepackage[T1]{fontenc}

\def\Name{Zackery Field}  % Your name
\def\Sec{Di, 103}  % Your GSI's name and discussion section
\def\Login{cs170-fe} % Your login
\def\Homework{7}%Number of Homework, PUT SOMETHING HERE
\def\Session{Fall 2013}

\title{CS170  Fall 2013 Solutions to Homework 7}
\author{\Name, section \Sec, \texttt{\Login}}
\markboth{CS170 --\Session\  Homework \Homework\ \Name, section \Sec}
{CS170--\Session\ Homework \Homework\ \Name, section \Sec, \texttt{\Login}}
\pagestyle{myheadings}

\begin{document}
\maketitle

\section*{1. (20 pts.) A greedy algorithm - so to speak}

We can formalize this as a graph problem. Let the undirected graph
 $G = (V,E)$ denote LinkedIn's relationship graph, where each vertex represents a person x
who has an account on LinkedIn. 
There is an edge $\{u, v\} \in E$ if $u$ and $v$ have listed a professional 
relationship with each other on LinkedIn (we will assume that relationships are symmetric). 
We are looking for a subset $S \subseteq V$ of vertices so that every vertex $s \in S$
 has edges to at least 20 other vertices in $S$. 
And we want to make the set $S$ as large as possible, subject to these constraints.
Design an efficient algorithm to find the set of super-schmoozers (the largest set $S$
 that is consistent with these constraints), given the graph $G$.

We can begin by identifying any vertices that have degree $< 20$.  Let the set of these
vertices be called $\bar{S}$.
Since the definition of an edge $s\in S$ depends on its neighbors recursively, 
there has to be a way tracking the number degree of the edges that fall into $S$ and then
that information must be returned to all other nodes in the path. 

This can be accomplished with a DFS with ties being broken on back edges. Simply mark each 
edge with pre- and post-visit numberings. As you run the search you will run into three cases:

\begin{itemize}
\item You will arrive at a vertex $v$ that is a sink in the sense that it is connected to 
$< 20$ vertices $\in S$. By definition, $v \in \bar{S}$.

\item You wil arrive at a vertex $v$ that is connected to $\geq 20$ vertices $\in S$. 
In this case you can not be sure whether or not $v\in S$ because it is not yet known
whether the vertices connnected to it are $\in S$.

\item Arrival at a vertex $v$ reveals a back edge whose vertex $l$. If the relationship
$post(v) - pre(l) \geq 20$ holds, then the vertex $v$ is in a cycle of length $\geq 20$ 
containing vertices   
\end{itemize}

To be continued : sorry!
\label{pg:end-of-p1}
% Make sure that the solution here does not exceed one page here. If
% it does, use the extra space for this problem at the end.  
%
% Comment out the next line if you are NOT using the extra space
% \paragraph{} \emph{Continued on Page \pageref{pg:p1-continuation}}

\newpage

%%Do NOT remove/comment the next line
\pagestyle{plain}
%%It makes sure your name appears only on the first page

\section*{2. (20 pts.) A funky kind of coloring}

Let $G = (V,E)$ be an undirected graph where every vertex has degree $\leq 51$. 
Let's find a way of coloring each vertex blue or gold, so that no vertex
has more than 25 neighbors of its own color.

Consider the following algorithm, where we call a vertex `bad` if it
has more than 25 neighbors of its own color:

\begin{itemize}  
\item[{\bf 1.}] Color each vertex arbitrarily 
\item[{\bf 2.}] Let $B := \{v \in V : v \mbox{ is bad}\}$
\item[{\bf 3.}] While $B \not= \emptyset:$
\item[{\bf 4.}] \hspace{4 pt}    Pick any bad vertex $v \in B$
\item[{\bf 5.}] \hspace{4 pt}    Reverse the color of $v$.
\item[{\bf 6.}] \hspace{4 pt}    Update $B$ to reflect this chang,
so that it again holds the set of bad vertices.
\end{itemize}

Notice that if the algorithm terminates then, it is 
guranteed to find a coloring with the desired property.

\begin{itemize}
\item[{\bf (a)}] Prove that this algorithm terminates in a finite number of steps. 
I suggest that you define a \emph{potential function} that associates a non-negative
integer (the potential) to each possible way of coloring the graph, in such a way that
each interation of the while-loop is guranteed to strictly reduce the potential.

Following the hint, allow $p(G):G\mapsto\mathbb{N}_0$ to denote the potential of a graph $G$.
 Let this potential be explicitly defined as the number of edges in $E$ that have the same color
vertex on either end (same-color edges), which by definition always takes on a value $\in \mathbb{N}_0$. 
Each time that the loop runs, a vertex $v\in B$ is changed from 
one color to the other.  $\forall v \in B$ there are $> 25$ vertices adjacent to $v$ and of the same color,
and there are at most $25$ vertices adjacent to $v$ of the opposite color. 
This shows that even with the maximal number of non-similar and minimal similar adjacent 
vertices there is still a decrease in the number of same-color edges  by at least one (namely, one 
of the edges with $v$ as an end)
when you switch the color of $v$. Thus, each iteration of the loop leads to a decrease of
the value of $p(G)$ by \emph{at least} one.

\item[{\bf (b)}] Prove that the algorithm terminates after at most $|E|$ iterations
of the loop. Hint: You should figure out the largest value the potential could take on.

If $\forall v \in V: v\in B$ then  the potential function $p(G)$ described above
takes on its maximum value, $p(G)= |E|$.  It was shown in (a) that for each iteration of the loop
for a chosen $v \in B$ the decrease in $p(G)$ is at least one. Therefore, even in this worst case 
scenario, the algorithm will terminate after $|E|$ iterations.

% For this graph, began at any edge $v$ maintain its color, and
% label this edge as being at level $0$.Travel along each of the edges of $v$ and mark 
% all of the vertices with the opposite color of $v$ and label each of these edges as being at 
% level $1$. Continue this process until you either reach a sink or the edges all point back to
% a vertex that has already been repainted. Continue to paint all the even levels one color and 
% all of the odd levels the other color. There is no case where there 

\end{itemize}

% Optional: Think about how to implement the algorithm so that its total running time 
% is $O(|V|+|E|)$ 

\label{pg:end-of-p2}
% Make sure that the solution here does not exceed one page here. If
% it does, use the extra space for this problem at the end.  
%
% Comment out the next line if you are NOT using the extra space
% \paragraph{} \emph{Continued on Page \pageref{pg:p2-continuation}}

\newpage

\section*{3. (20 pts.) Job Scheduling}

You are given a set of $n$ jobs, each runs in unit time.
 Job $i$ has an integer-valued deadline time $d_i \geq  0$ and a real-valued penalty $p_i \geq 0$.
 Jobs may be scheduled to start at any non-negative integer time $(0, 1, 2, etc)$, and
 only one job may run at a time. If job $i$ completes at or before time $d_i$, then it 
incurs no penalty; otherwise, it is late and incurs penalty $p_i$. The goal is to 
schedule all jobs so as to minimize the total penalty incurred.
For each of the following greedy algorithms, either prove that it is correct, or give
 a simple counterexample (with at most three jobs) to show that it fails.

\begin{itemize}
 
\item[{\bf (a)}] Among the unscheduled jobs that can be scheduled on time, consider the
one whose deadline is the earliest (breaking ties with the highest penalty), and schedule 
it at the earliest available time. Repeat.

counterexample: 
\begin{equation*}
  \begin{align}
    j_a&=\{d_a&=1,p_a&=5\}\\
    j_b&=\{d_b&=2,p_b&=15\}\\
    j_c&=\{d_c&=2,p_c&=10\}\\
  \end{align}
\end{equation*}

The minimized total penalty for this set is: $p=5$, where $j_b$ and $j_c$ are compelted. 

However, the algorithm at time $t=0$ will schedule job $j_a$ because it has the earliest
deadline, and then it will schedule $j_b$ at time $t=1$ breaking ties with the highest
penalty. This leaves the total incurred penalty as being $p_{total}=10$.

\item[{\bf (b)}] Among unscheduled jobs that can be scheduled on time, consider the one
 whose penalty is the highest (breaking ties with the earliest deadline), and schedule
 it at the earliest available time. Repeat.

counterexample: 
\begin{equation*}
  \begin{align}
    j_a&=\{d_a&=1,p_a&=5\}\\
    j_b&=\{d_b&=2,p_b&=10\}\\
    j_c&=\{d_c&=3,p_c&=15\}\\
  \end{align}
\end{equation*}

The minimal penalty incurred for this example is: $p=0$, with all jobs being completed.

For this algorithm, the scheduling is based on penalty so the order of scheduling will be
$j_c$ start@ $t=0$, $j_b$ start@ $t=1$, which will incur a penalty of $p=5$.

\end{itemize}

\label{pg:end-of-p3}

% Make sure that the solution here does not exceed one page here. If
% it does, use the extra space for this problem at the end.  
%
% Comment out the next line if you are NOT using the extra space
 \paragraph{} \emph{Continued on Page \pageref{pg:p3-continuation}}

\newpage

\section*{4. (20 pts.) Timesheets}

Suppose we have $N$ jobs labelled $1,...,N$. For each job, you have determined the
bonus of completing the job, $V_i \geq 0$, a penalty per day that you accumulate
for not doing the job, $P_i \geq 0$, and the days required for you to successfully 
complete the job $R_i > 0$.

Every day, we choose one unfinished job to work on. A job i has been finished if we
 have spent $R_i$ days working on it. This doesn't necessarily mean you have to spend
 $R_i$ contiguous sequence of days working on job $i$. We start on day 1, and we want
 to complete all our jobs and finish with maximum reward. If we finish job $i$ at the
 end of day $t$, we will get reward $V_i - (tP_i)$. Note, this value can be negative 
if you choose to delay a job for too long.

Given this information, what is the optimal job scheduling policy to complete all of the jobs?

The objective function for N jobs that we are trying to maximize is:
\begin{equation*}
  \begin{align}
    reward(N)=\sum_{i=1}^{N}{V_i-(tP_i)}
  \end{align}
\end{equation*}

The optimal algorithm involves solving a linear system that maximizes this reward function.
An approach that is inspired by problem 3(c) would be to take each job that has the lowest value to penalty ratio
and schedule it for as late as possible with respect to maximizing the value of $V-(tP)$.
\label{pg:end-of-p4}

% Make sure that the solution here does not exceed one page here. If
% it does, use the extra space for this problem at the end.  
%
% Comment out the next line if you are NOT using the extra space
%\paragraph{} \emph{Continued on Page \pageref{pg:p4-continuation}}


\newpage

\section*{5. (20  pts.) Weighted Set Cover}

In class we looked at a greedy algorithm to solve the \emph{set cover} problem,
and proved that if the optimal set cover has size \emph{k}, then our greedy 
algorithm will find a set cover of size at most $k\log_e{n}$.

Here is a generalization of the set cover problem. 

\begin{itemize}
\item \emph{Input:} A set of elements $B$ of size $n$; sets $S_1,\ldots,S_m \subseteq B$;
positive weights $w_i,\ldots,w_m$.

\item \emph{Output:} A selection of the sets $S_i$ who union is $B$.

\item \emph{Cost:} The sum of the weights $w_i$ for the sets that were picked.

\end{itemize}

Design an algorithm to find the set cover with approximately the smallest cost.
Prove that if there is a solution with cost $k$, then your algorithm will find a solution 
with cost $O(k\log n)$.

\label{pg:end-of-p5}


% Make sure that the solution here does not exceed one page here. If
% it does, use the extra space for this problem at the end.  
%
% Comment out the next line if you are NOT using the extra space
\paragraph{} \emph{Continued on Page \pageref{pg:p5-continuation}}


\newpage

% 
%  \section*{Problem 6}
% 
% \label{pg:end-of-p6}
% 
% % Make sure that the solution here does not exceed one page here. If
% % it does, use the extra space for this problem at the end.  
% %
% % Comment out the next line if you are NOT using the extra space
% \paragraph{} \emph{Continued on Page \pageref{pg:p6-continuation}}
% 
% \newpage
% 
% 
% Comment out the "extra spaces" completely for the problems for you
% don't neethem
% 
% \section*{Extra space for Problem1}
% \emph{Continued from Page \pageref{pg:end-ofp1}}
% \label{pgcontinuation}
% 
% 
% 
% newpage
% %%Comment out the above three lines if you are not using extraspace
% %%for this problem.
% 
% 
% \section*{Extra space for Problem 2}
% \emph{Continued from Page \pageref{pg:end-of-p2}}
% 
% %Insert solution here
% 
% \label{pg:p2-continuation}
% \newpage
% %%Comment out the above three lines if you are not using extra space
% %%for this problem.
% 
% \section*{Extra space for Problem 3}
\emph{Continued from Page \pageref{pg:end-of-p3}}

%Insert solution here

\begin{itemize}
\item[{\bf (c)}] Among unscheduled jobs that can be scheduled on time, consider the one
 whose penalty is the highest (breaking ties arbitrarily), and schedule it at the latest
 available time before its deadline. Repeat.

For the highest penalty jobset $J'$ with deadline $t=i$, the time slots that the jobs will 
be scheduled in are $t=i-1$ to $t=i-|J'|$ (arbitrarily). The only case that is of any concern 
is when there is another set of jobs $J$ with deadlines within the timespan $t=\{i-1,i-|J'|\}$ and
maximized penalty s.t. penalty($J$) < penalty($J'$).
In this case, if there there was a more optimal solution that involved a job $j\in J$ then this 
job would belong to $J'$ and would have already been scheduled.

% Proof: By induction, there is a base case of a set of highest penalty jobs $J$ due at time $t=i$.
% Of these jobs, they will arbitrarily be scheduled at times $t=i-1$ to time $t=i-|J|$.
% The base case will then always make the optimal choice. 
% 
% The strong inductive hypothesis is that for some job set $J'$, all of the jobsets with penalty 
% greater than penalty($J'$) the optimal scheduling was chosen. The inductive step is to show that 
% the maximal jobset with penalty $< J'$ will provide an optimal solution. 
\end{itemize}
\label{pg:p3-continuation}

\newpage

%%Comment out the above three lines if you are not using extra space
%%for this problem.


% 
% \section*{Extra space for Problem 4}
% \emph{Continued from Page \pageref{pg:end-of-p4}}
% 
% 
% \label{pg:p4-continuation}
% \newpage
% %%Comment out the above three lines if you are not using extra space
% %%for this problem.
% 

% 
% \section*{Extra space for Problem 5}
% \emph{Continued from Page \pageref{pg:end-of-p5}}
% 
% \label{pg:p5-continuation}
% \newpage
% %%Comment out the above three lines if you are not using extra space
% %%for this problem.
% 
% 
% \section*{Extra space for Problem 6}
% \emph{Continued from Page \pageref{pg:end-of-p6}}
% 
% 
% %Insert solution here
% 
% 
% \label{pg:p6-continuation}
% \newpage
% %%Comment out the above three lines if you are not using extra space
% %%for this problem.
% 


\end{document}

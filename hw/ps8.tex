\documentclass[11pt]{article}
\usepackage{amsmath,textcomp,amssymb,geometry,graphicx,tikz,cancel}
\usepackage{algpseudocode,algorithm}
\usepackage[T1]{fontenc}

\def\Name{Zackery Field}  % Your name
\def\Sec{Di, 103}  % Your GSI's name and discussion section
\def\Login{cs170-fe} % Your login
\def\Homework{8}%Number of Homework
\def\Session{Fall 2013}

\title{CS170  Fall 2013 Solutions to Homework 8}
\author{\Name, section \Sec, \texttt{\Login}}
\markboth{CS170 --\Session\  Homework \Homework\ \Name, section \Sec}
{CS170--\Session\ Homework \Homework\ \Name, section \Sec, \texttt{\Login}}
\pagestyle{myheadings}

\begin{document}
\maketitle

\section*{1. 6.4}

You are given a string of $n$ characters $s[1,\ldots,n]$, which you believe to be a corrupted text document
in which all punctuation has vanished. You wish to reconstruct the document using a dictionary, 
which is available in the form of a Boolean function $dict(\dot)$: for any string $w$,

\begin{displaymath}
   dict(w) = \left\{
     \begin{array}{lr}
       true  & \mbox{if $w$ is a valid word}\\
       false & \mbox{otherwise}
     \end{array}
   \right.
\end{displaymath} 

\begin{itemize}

\item[{\bf (a)}] Give a dynamic programming algorithm that determines whether the string $s$ can be reconstructed as a sequence of valid words.
The running time should be at most $O(n^2)$, assuming that the calls to dict take unit time.

The smallest substring $w_s$ that will yield $dict(w) = true$ is of length $1$.
We can break $s$ into subproblems of $\{s_i,s_{i+1},\ldots,s_j\}$ where $i\leqj$. 


We can discover whether each of these substrings is a word in $O(1)$ time, and there are $n^2$ such problems so the
total running time is $O(n^2)$.

\item[{\bf (b)}] In the event that the string is valid, make your algorithm output the corresponding sequence of words.

\end{itemize}
\label{pg:end-of-p1}
% Make sure that the solution here does not exceed one page here. If
% it does, use the extra space for this problem at the end.  
%
% Comment out the next line if you are NOT using the extra space
 \paragraph{} \emph{Continued on Page \pageref{pg:p1-continuation}}

\newpage

%%Do NOT remove/comment the next line
\pagestyle{plain}
%%It makes sure your name appears only on the first page

\section*{2. 6.14 Cutting Cloth}

You are given a rectangular piece of cloth with dimensions $X\times Y$, where $X$ and $Y$ are positive integers, and a list of $n$ products that can be made using the cloth. 
For each product $i \in [1,n]$ you know that a rectangle of cloth of dimensions $a_i \times b_i$ is needed and that the final selling price of the proucdt is $c_i$. Assume that $a_i$, $b_i$, and $c_i$ are all positive integers.
You have a machine that can cut any rectangular piece of cloth into two pieces either horizontally or vertically .
Design an efficient dynamic programming solution that determines the best strategy for cutting the $X\times Y$ piece of cloth, that is, a strategy for cutting the cloth so that the products made fomr the resulting pieces give the maximum sum of selling prices.
You are free to make as many copies of a given product as you wish, or none if desired.

\label{pg:end-of-p2}
% Make sure that the solution here does not exceed one page here. If
% it does, use the extra space for this problem at the end.  
%
% Comment out the next line if you are NOT using the extra space
 \paragraph{} \emph{Continued on Page \pageref{pg:p2-continuation}}

\newpage

\section*{3. 6.20 Optimal binary search trees}

Suppose we know the frequency with which keywords occur in programs of a certain language.
We want to organize them in a binary search tree, so that the keyword in the root is alphabetically bigger than all the keywords in the left subtree and smaller than all the keywords in the right subtree.

Figure 6.12 has a nicely-balanced example on the left. In this case, when a keyword is being looked up, the number of comparisons needed is at most three: for instance finding while, only the three nodes `end', 'then', and 'while' get examined.
But since we know the frequency with which keywords are accessed, we can use an even more fine-tuned cost function, the average number of comparisons to look up a word: 

$$ cost = lvl_1(w_{1_1}+\ldots+w_{1_n}) + \cdots + lvl_n(w_{n_1}+\cdots+w_{n_n})$$

By this measure, the best search tree is the one on the right, which has a cost of $2.18$. 

Give an efficient algorithm for the following task:

\emph{Input:} $n$ words (in sorted order); frequencies of these words: $p_1,p_2,\ldots,p_n$.

\emph{Output:} The binary search tree of lowest cost.


\label{pg:end-of-p3}

% Make sure that the solution here does not exceed one page here. If
% it does, use the extra space for this problem at the end.  
%
% Comment out the next line if you are NOT using the extra space
 \paragraph{} \emph{Continued on Page \pageref{pg:p3-continuation}}

\newpage

\section*{4. 6.29 Exon chaining}

Each gene corresponds to a subregion of an overall genome (the DNA sequence); however
, part of this region might be `junk DNA'. Frequently, gene consists of 
several pieces called exons, seperated by junk fragments called introns. This complicates
the process of identifying genes in a newly sequenced genome.

Suppose we have a new DNA sequence and we want to check whether a certain gene (a string) 
is present in it. Because we cannot hope that the gene will be a contiguous subsequence,
we look for partial matches-fragments of DNA that are also present in the gene (actually, 
even these partial matches will be approximate, not perfect). We then attempt to assemble 
these fragments. 

Let $x[1,\ldots,n]$ denote the DNA sequence. Each partial match can be represented by a triple 
$(l_i,r_i,w_i)$, where $x[l_i,\ldots,r_i]$ is the fragment and $w_i$ is the weight representing 
the strength of the match (it might be a local alignment score or some other statistical 
quantity). Many of these potential matches could be false, so the goal is to find a subset
of the triples that are consistent (nonoverlapping) and have a maximum total weight.

Show how to do this efficiently.


\label{pg:end-of-p4}

% Make sure that the solution here does not exceed one page here. If
% it does, use the extra space for this problem at the end.  
%
% Comment out the next line if you are NOT using the extra space
\paragraph{} \emph{Continued on Page \pageref{pg:p4-continuation}}


\newpage

\section*{5. Timesheet Part 2}

Recall problem 4 from homework 7.
Suppose we have $N$ jobs labelled $1,...,N$. For each job, you have determined the bonus of
 completing the job, $V_i ≥ 0$, a penalty per day that you accumulate for not doing the job,
$ P_i ≥ 0$, and the days required for you to successfully complete the job $R_i > 0$.

Every day, we choose one unfinished job to work on. A job $i$ has been finished if we have
 spent $R_i$ days working on it. This doesn’t necessarily mean you have to spend $R_i$ contiguous
 sequence of days working on job $i$. We start on day $1$, and we want to complete all our 
jobs and finish with maximum reward. If we finish job $i$ at the end of day $t$, we will get
 reward $V_i −t\timesPi$. Note, this value can be negative if you choose to delay a job for too long.

Now, what we did not tell you last time is that we have a time limit of $T$ days, in which we can
 choose to work on some of these jobs in only those $T$ days. Given this information, what is the optimal
 job scheduling policy with a time limit of $T$ days? 
Notice that 0 is a lower bound since we can choose to do no jobs at all if all of them 
happen to have negative value, or all of them take more than time $T$.

Design an efficient dynamic programming algorithm that finds the optimal schedule.

\label{pg:end-of-p5}


% Make sure that the solution here does not exceed one page here. If
% it does, use the extra space for this problem at the end.  
%
% Comment out the next line if you are NOT using the extra space
\paragraph{} \emph{Continued on Page \pageref{pg:p5-continuation}}

\newpage

% 
%  \section*{Problem 6}
% 
% \label{pg:end-of-p6}
% 
% % Make sure that the solution here does not exceed one page here. If
% % it does, use the extra space for this problem at the end.  
% %
% % Comment out the next line if you are NOT using the extra space
% \paragraph{} \emph{Continued on Page \pageref{pg:p6-continuation}}
% 
% \newpage
% 
% 
% Comment out the "extra spaces" completely for the problems for you
% don't neethem
 
 \section*{Extra space for Problem 1}
 \emph{Continued from Page \pageref{pg:end-of-p1}}
 

 \label{pg:p1-continuation}
 
 
 
 \newpage
 %%Comment out the above three lines if you are not using extraspace
 %%for this problem.
 
 
 \section*{Extra space for Problem 2}
 \emph{Continued from Page \pageref{pg:end-of-p2}}
 
 %Insert solution here
 
 \label{pg:p2-continuation}
 \newpage
 %%Comment out the above three lines if you are not using extra space
 %%for this problem.
 
 \section*{Extra space for Problem 3}
 \emph{Continued from Page \pageref{pg:end-of-p3}}
 
 \label{pg:p3-continuation}

\newpage
%Comment out the above three lines if you are not using extra space
%for this problem.
 
 \section*{Extra space for Problem 4}
 \emph{Continued from Page \pageref{pg:end-of-p4}}
 
 
 \label{pg:p4-continuation}
 \newpage
 %%Comment out the above three lines if you are not using extra space
 %%for this problem.
 
 
 \section*{Extra space for Problem 5}
 \emph{Continued from Page \pageref{pg:end-of-p5}}
 
 \label{pg:p5-continuation}
 \newpage
 %%Comment out the above three lines if you are not using extra space
 %%for this problem.
 
 
% \section*{Extra space for Problem 6}
% \emph{Continued from Page \pageref{pg:end-of-p6}}
% 
% 
% %Insert solution here
% 
% 
% \label{pg:p6-continuation}
% \newpage
% %%Comment out the above three lines if you are not using extra space
% %%for this problem.
% 


\end{document}
